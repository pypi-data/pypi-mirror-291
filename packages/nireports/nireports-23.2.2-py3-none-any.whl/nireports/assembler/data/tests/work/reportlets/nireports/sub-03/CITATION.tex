% Options for packages loaded elsewhere
\PassOptionsToPackage{unicode}{hyperref}
\PassOptionsToPackage{hyphens}{url}
%
\documentclass[
]{article}
\usepackage{amsmath,amssymb}
\usepackage{lmodern}
\usepackage{iftex}
\ifPDFTeX
  \usepackage[T1]{fontenc}
  \usepackage[utf8]{inputenc}
  \usepackage{textcomp} % provide euro and other symbols
\else % if luatex or xetex
  \usepackage{unicode-math}
  \defaultfontfeatures{Scale=MatchLowercase}
  \defaultfontfeatures[\rmfamily]{Ligatures=TeX,Scale=1}
\fi
% Use upquote if available, for straight quotes in verbatim environments
\IfFileExists{upquote.sty}{\usepackage{upquote}}{}
\IfFileExists{microtype.sty}{% use microtype if available
  \usepackage[]{microtype}
  \UseMicrotypeSet[protrusion]{basicmath} % disable protrusion for tt fonts
}{}
\makeatletter
\@ifundefined{KOMAClassName}{% if non-KOMA class
  \IfFileExists{parskip.sty}{%
    \usepackage{parskip}
  }{% else
    \setlength{\parindent}{0pt}
    \setlength{\parskip}{6pt plus 2pt minus 1pt}}
}{% if KOMA class
  \KOMAoptions{parskip=half}}
\makeatother
\usepackage{xcolor}
\setlength{\emergencystretch}{3em} % prevent overfull lines
\providecommand{\tightlist}{%
  \setlength{\itemsep}{0pt}\setlength{\parskip}{0pt}}
\setcounter{secnumdepth}{-\maxdimen} % remove section numbering
\ifLuaTeX
  \usepackage{selnolig}  % disable illegal ligatures
\fi
\usepackage[]{natbib}
\bibliographystyle{plainnat}
\IfFileExists{bookmark.sty}{\usepackage{bookmark}}{\usepackage{hyperref}}
\IfFileExists{xurl.sty}{\usepackage{xurl}}{} % add URL line breaks if available
\urlstyle{same} % disable monospaced font for URLs
\hypersetup{
  hidelinks,
  pdfcreator={LaTeX via pandoc}}

\author{}
\date{}

\begin{document}

Results included in this manuscript come from preprocessing performed
using \emph{fMRIPrep} 22.0.0 (\citet{fmriprep1}; \citet{fmriprep2};
RRID:SCR\_016216), which is based on \emph{Nipype} 1.8.4.dev0
(\citet{nipype1}; \citet{nipype2}; RRID:SCR\_002502).

\begin{description}
\item[Preprocessing of B0 inhomogeneity mappings]
A total of 1 fieldmaps were found available within the input BIDS
structure for this particular subject. A \emph{B0} nonuniformity map (or
\emph{fieldmap}) was estimated from the phase-drift map(s) measure with
two consecutive GRE (gradient-recalled echo) acquisitions. The
corresponding phase-map(s) were phase-unwrapped with \texttt{prelude}
(FSL 6.0.4:ddd0a010).
\item[Anatomical data preprocessing]
A total of 1 T1-weighted (T1w) images were found within the input BIDS
dataset.The T1-weighted (T1w) image was corrected for intensity
non-uniformity (INU) with \texttt{N4BiasFieldCorrection} \citep{n4},
distributed with ANTs 2.3.3 \citep[RRID:SCR\_004757]{ants}, and used as
T1w-reference throughout the workflow. The T1w-reference was then
skull-stripped with a \emph{Nipype} implementation of the
\texttt{antsBrainExtraction.sh} workflow (from ANTs), using OASIS30ANTs
as target template. Brain tissue segmentation of cerebrospinal fluid
(CSF), white-matter (WM) and gray-matter (GM) was performed on the
brain-extracted T1w using \texttt{fast} \citep[FSL 6.0.4:ddd0a010,
RRID:SCR\_002823,][]{fsl_fast}. Brain surfaces were reconstructed using
\texttt{recon-all} \citep[FreeSurfer 7.2.0,
RRID:SCR\_001847,][]{fs_reconall}, and the brain mask estimated
previously was refined with a custom variation of the method to
reconcile ANTs-derived and FreeSurfer-derived segmentations of the
cortical gray-matter of Mindboggle
\citep[RRID:SCR\_002438,][]{mindboggle}. Volume-based spatial
normalization to one standard space (MNI152NLin2009cAsym) was performed
through nonlinear registration with \texttt{antsRegistration} (ANTs
2.3.3), using brain-extracted versions of both T1w reference and the T1w
template. The following template was selected for spatial normalization:
\emph{ICBM 152 Nonlinear Asymmetrical template version 2009c}
{[}\citet{mni152nlin2009casym}, RRID:SCR\_008796; TemplateFlow ID:
MNI152NLin2009cAsym{]}.
\item[Functional data preprocessing]
For each of the 4 BOLD runs found per subject (across all tasks and
sessions), the following preprocessing was performed. First, a reference
volume and its skull-stripped version were generated using a custom
methodology of \emph{fMRIPrep}. Head-motion parameters with respect to
the BOLD reference (transformation matrices, and six corresponding
rotation and translation parameters) are estimated before any
spatiotemporal filtering using \texttt{mcflirt} \citep[FSL
6.0.4:ddd0a010,][]{mcflirt}. The estimated \emph{fieldmap} was then
aligned with rigid-registration to the target EPI (echo-planar imaging)
reference run. The field coefficients were mapped on to the reference
EPI using the transform. BOLD runs were slice-time corrected to 0.972s
(0.5 of slice acquisition range 0s-1.94s) using \texttt{3dTshift} from
AFNI 20210200 \citep[RRID:SCR\_005927]{afni}. The BOLD reference was
then co-registered to the T1w reference using \texttt{bbregister}
(FreeSurfer) which implements boundary-based registration \citep{bbr}.
Co-registration was configured with six degrees of freedom. Several
confounding time-series were calculated based on the \emph{preprocessed
BOLD}: framewise displacement (FD), DVARS and three region-wise global
signals. FD was computed using two formulations following Power
(absolute sum of relative motions, \citet{power_fd_dvars}) and Jenkinson
(relative root mean square displacement between affines,
\citet{mcflirt}). FD and DVARS are calculated for each functional run,
both using their implementations in \emph{Nipype} \citep[following the
definitions by][]{power_fd_dvars}. The three global signals are
extracted within the CSF, the WM, and the whole-brain masks.
Additionally, a set of physiological regressors were extracted to allow
for component-based noise correction \citep[\emph{CompCor},][]{compcor}.
Principal components are estimated after high-pass filtering the
\emph{preprocessed BOLD} time-series (using a discrete cosine filter
with 128s cut-off) for the two \emph{CompCor} variants: temporal
(tCompCor) and anatomical (aCompCor). tCompCor components are then
calculated from the top 2\% variable voxels within the brain mask. For
aCompCor, three probabilistic masks (CSF, WM and combined CSF+WM) are
generated in anatomical space. The implementation differs from that of
Behzadi et al.~in that instead of eroding the masks by 2 pixels on BOLD
space, a mask of pixels that likely contain a volume fraction of GM is
subtracted from the aCompCor masks. This mask is obtained by dilating a
GM mask extracted from the FreeSurfer's \emph{aseg} segmentation, and it
ensures components are not extracted from voxels containing a minimal
fraction of GM. Finally, these masks are resampled into BOLD space and
binarized by thresholding at 0.99 (as in the original implementation).
Components are also calculated separately within the WM and CSF masks.
For each CompCor decomposition, the \emph{k} components with the largest
singular values are retained, such that the retained components' time
series are sufficient to explain 50 percent of variance across the
nuisance mask (CSF, WM, combined, or temporal). The remaining components
are dropped from consideration. The head-motion estimates calculated in
the correction step were also placed within the corresponding confounds
file. The confound time series derived from head motion estimates and
global signals were expanded with the inclusion of temporal derivatives
and quadratic terms for each \citep{confounds_satterthwaite_2013}.
Frames that exceeded a threshold of 0.5 mm FD or 1.5 standardized DVARS
were annotated as motion outliers. Additional nuisance timeseries are
calculated by means of principal components analysis of the signal found
within a thin band (\emph{crown}) of voxels around the edge of the
brain, as proposed by \citep{patriat_improved_2017}. The BOLD
time-series were resampled into standard space, generating a
\emph{preprocessed BOLD run in MNI152NLin2009cAsym space}. First, a
reference volume and its skull-stripped version were generated using a
custom methodology of \emph{fMRIPrep}. All resamplings can be performed
with \emph{a single interpolation step} by composing all the pertinent
transformations (i.e.~head-motion transform matrices, susceptibility
distortion correction when available, and co-registrations to anatomical
and output spaces). Gridded (volumetric) resamplings were performed
using \texttt{antsApplyTransforms} (ANTs), configured with Lanczos
interpolation to minimize the smoothing effects of other kernels
\citep{lanczos}. Non-gridded (surface) resamplings were performed using
\texttt{mri\_vol2surf} (FreeSurfer).
\end{description}

Many internal operations of \emph{fMRIPrep} use \emph{Nilearn} 0.9.1
\citep[RRID:SCR\_001362]{nilearn}, mostly within the functional
processing workflow. For more details of the pipeline, see
\href{https://fmriprep.readthedocs.io/en/latest/workflows.html}{the
section corresponding to workflows in \emph{fMRIPrep}'s documentation}.

\hypertarget{references}{%
\subsubsection{References}\label{references}}

  \bibliography{/data/derivatives/ds002790/fmriprep-dev/logs/CITATION.bib}

\end{document}
