\documentclass{article}
% common styling and macros shared by all proof files

\usepackage[top=1in, right=1in, left=1in, bottom=1.5in]{geometry}

\usepackage{amsmath,amsthm,amsfonts,amssymb,amscd}
\usepackage{listings}
\usepackage{hyperref}
\usepackage{xcolor}
\usepackage{xr}

\usepackage{enumerate} 
\usepackage{physics}
\usepackage{fancyhdr}
\usepackage{hyperref}
\usepackage{graphicx}
\usepackage{tcolorbox}
\usepackage{catchfile}
\usepackage{pdftexcmds}
\usepackage[T1]{fontenc}

% hyperref
\hypersetup{
  colorlinks=true,
  linkcolor=blue,
  linkbordercolor={0 0 1}
}

% \contrib macro to indicate inclusion in "contrib".
\usepackage{tcolorbox}
\newtcolorbox{warn_box}{colback=red!5!white,colframe=red!75!black}
\newcommand{\contrib}{{\begin{warn_box}This proof resides in \textbf{``contrib''} because it has not completed the vetting process.\end{warn_box}}} 
\newcommand{\floatingPoint}{{\begin{warn_box}This implementation is susceptible to floating-point vulnerabilities.\end{warn_box}}} 

% asOfCommit macro to version a code dependency. Arguments:
%    #1: relative path to file you are dependent on
%    #2: commit hash it was last edited. If outdated, this should be the second hash in the footnoote. Otherwise,
%            git log -n 1 --pretty=format:%h -- path/to/file.rs
\makeatletter
\ifnum\pdf@shellescape=1
   % "private" command that builds a link to a blob
  \newcommand{\linkOpendpBlob}[3]{%
    \href{https://github.com/opendp/opendp/blob/#1/#2#3}{\path{#3} at commit #1}}

  % latex macro expansion has a separate phase for \input evaluation
  %     immediately evaluate a command to write a temp file to ./out containing the current directory
  \immediate\write18{[ ! -f out/cwd.txt ] && (mkdir -p out && git rev-parse --show-prefix | sed "s|_|\@backslashchar\@backslashchar\@backslashchar_|g" > out/cwd.txt)}
  %     ...and then retrieve the current working directory by loading the temp file
  \CatchFileDef\GitWorkingDir{out/cwd.txt}{\endlinechar=-1}

  % command for building the (up to date) or (outdated) status
  \newcommand{\fileStatus}[2]{%
  \setbox0=\hbox{\input|"git --no-pager log -n1 --pretty='\@percentchar H' #1 | grep -E '^#2.*'"\unskip}\ifdim\wd0=0pt
        (outdated\footnote{See new changes with \texttt{git diff #2..\input|"git --no-pager log -n1 --pretty='\@percentchar h' #1" \GitWorkingDir\path{#1}}})\else
        (up to date)\fi
  }

  \newcommand{\asOfCommit}[2]{%
      % permalink the target
      \linkOpendpBlob{#2}{\GitWorkingDir}{#1}
      % conditionally add (outdated) or (up to date) depending on matching commit hash
      \fileStatus{#1}{#2}%
  }
\else
  % simplified command if shell-escape not enabled
  \newcommand{\asOfCommit}[2]{#1 at commit #2 (unknown status\footnote{Shell-escape is not enabled. Enable \texttt{--shell-escape} to check if this proof is up-to-date with the code.})}
\fi
\makeatother

% \vettingPR macro to link a PR. Arguments:
%    #1: PR number
\newcommand{\vettingPR}[1]{\href{https://github.com/opendp/opendp/pull/#1}{Pull Request \##1}}

% for links to rustdoc items in OpenDP. Arguments:
%    #1: path to item, and designation as trait, struct, fn, etc.
%    #2: item name
\makeatletter
\ifnum\pdf@shellescape=1
  % latex macro expansion has a separate phase for \input evaluation
  %     immediately evaluate a command to write a temp file to ./out containing the base path
  \immediate\write18{[ ! -f out/rustdoc.txt ] && mkdir -p out && ([ -z `kpsewhich --var-value OPENDP_RUSTDOC_PORT` ] && echo "https://docs.rs/opendp/`head -n 1 \@backslashchar`git rev-parse --show-toplevel\@backslashchar`/VERSION | sed 's|.*-dev.*|latest|g'`" || echo "http://localhost:`kpsewhich --var-value OPENDP_RUSTDOC_PORT`") > out/rustdoc.txt}
  %     ...and then retrieve the base path by loading the temp file
  \CatchFileDef\OpenDPRustdocBase{out/rustdoc.txt}{\endlinechar=-1}
\else
  % if shell commands are not enabled, just claim latest
  \newcommand{\OpenDPRustdocBase}{https://docs.rs/opendp/latest}
\fi
\makeatother
\newcommand{\rustdoc}[2]{\href{\OpenDPRustdocBase/opendp/#1.#2.html}{\texttt{#2}}}

% for links to external dependencies. Arguments:
%    #1: crate name
%    #2: path to item, and designation as trait, struct, fn, etc.
%    #3: item name
\newcommand{\docsrs}[3]{\href{https://docs.rs/#1/latest/#1/#2.#3.html}{\texttt{#3}}}

% minted (pseudocode)
\definecolor{codegreen}{rgb}{0,0.6,0}
\definecolor{codegray}{rgb}{0.5,0.5,0.5}
\definecolor{codepurple}{rgb}{0.58,0,0.82}
\definecolor{backcolour}{rgb}{0.95,0.95,0.92}

\lstdefinestyle{mystyle}{
    backgroundcolor=\color{backcolour},   
    commentstyle=\color{codegreen},
    keywordstyle=\color{magenta},
    numberstyle=\tiny\color{codegray},
    stringstyle=\color{codepurple},
    basicstyle=\ttfamily\footnotesize,
    breakatwhitespace=false,         
    breaklines=true,                 
    captionpos=b,                    
    keepspaces=true,                 
    numbers=left,                    
    numbersep=5pt,                  
    showspaces=false,                
    showstringspaces=false,
    showtabs=false,                  
    tabsize=2
}

\lstset{style=mystyle}

% common commands
\theoremstyle{definition}
\newtheorem{theorem}{Theorem}[section]
\newtheorem{lemma}[theorem]{Lemma}
\newtheorem{definition}[theorem]{Definition}
\newtheorem{warning}{Warning}
\newtheorem{corollary}{Corollary}
\newtheorem{proposition}{Proposition}
\newtheorem{remark}{Remark}
\newtheorem{observation}{Observation}
\newtheorem{note}{Note}

\newcommand{\vicki}[1]{{ {\color{olive}{(vicki)~#1}}}}
\newcommand{\hanwen}[1]{{ {\color{purple}{(hanwen)~#1}}}}
\newcommand{\zach}[1]{{ {\color{red}{(zach)~#1}}}}

\newcommand{\MultiSet}{\mathrm{MultiSet}}
\newcommand{\len}{\mathrm{len}}
\newcommand{\din}{\texttt{d\_in}}
\newcommand{\dout}{\texttt{d\_out}}
\newcommand{\T}{\texttt{T} }
\newcommand{\F}{\texttt{F} }
\newcommand{\Map}{\texttt{Map}}
\newcommand{\X}{\mathcal{X}}
\newcommand{\Y}{\mathcal{Y}}
\newcommand{\True}{\texttt{True}}
\newcommand{\False}{\texttt{False}}
\newcommand{\clamp}{\texttt{clamp}}
\newcommand{\function}{\texttt{function}}
\newcommand{\float}{\texttt{float }}
\newcommand{\questionc}[1]{\textcolor{red}{\textbf{Question:} #1}}


\newcommand{\validTransformation}[2]{%
  \begin{theorem}
  For every setting of the input parameters #1 to #2 such that the given preconditions
  hold, #2 raises an exception (at compile time or run time) or returns a valid transformation. A valid transformation has the following properties:
  \begin{enumerate}
      \item \textup{(Appropriate output domain).} 
      For every element $x$ in \texttt{input\_domain}, $\function(x)$ is in \texttt{output\_domain} or raises a data-independent runtime exception.
      
      \item \textup{(Stability guarantee).} 
      For every pair of elements $x, x'$ in \texttt{input\_domain} and for every pair $(\din, \dout)$, 
      where \din\ has the associated type for \texttt{input\_metric} and \dout\ has the associated type for \\ \texttt{output\_metric}, 
      if $x, x'$ are \din-close under \texttt{input\_metric}, $\texttt{stability\_map}(\din)$ does not raise an exception,
      and $\texttt{stability\_map}(\din) \leq \dout$, 
      then $\function(x), \function(x')$ are $\dout$-close under \texttt{output\_metric}.
  \end{enumerate}
  \end{theorem}
}


\newcommand{\validMeasurement}[2]{%
  \begin{theorem}
  For every setting of the input parameters #1 to #2 such that the given preconditions
  hold, #2 raises an exception (at compile time or run time) or returns a valid measurement. A valid measurement has the following property:
  \begin{enumerate}
      \item \textup{(Privacy guarantee).}
      For every pair of elements $x, x'$ in \texttt{input\_domain} and for every pair $(\din, \dout)$,
      where \din\ has the associated type for \texttt{input\_metric} and \dout\ has the associated type for \\ \texttt{output\_measure},
      if $x, x'$ are \din-close under \texttt{input\_metric}, $\texttt{privacy\_map}(\din)$ does not raise an exception,
      and $\texttt{privacy\_map}(\din) \leq \dout$,
      then $\function(x), \function(x')$ are $\dout$-close under \texttt{output\_measure}.
  \end{enumerate}
  \end{theorem}
}


\title{\texttt{fn make\_private\_group\_by}}
\author{Michael Shoemate}
\date{}

\begin{document}

\maketitle

\contrib
Proves soundness of \texttt{fn make\_private\_group\_by} in \asOfCommit{mod.rs}{0db9c6036}.

\subsection*{Vetting History}
\begin{itemize}
    \item \vettingPR{512}
\end{itemize}

\section{Hoare Triple}
\subsection*{Precondition}
To ensure the correctness of the output, we require the following preconditions:

\begin{itemize}
    \item Type \texttt{MS} must have trait \texttt{DatasetMetric}.
    \item Type \texttt{MI} must have trait \texttt{UnboundedMetric}.
    \item Type \texttt{MO} must have trait \texttt{ApproximateMeasure}.
\end{itemize}

\subsection*{Pseudocode}
\lstinputlisting[language=Python,firstline=2,escapechar=|]{./pseudocode/make_private_group_by.py}

\subsubsection*{Postconditions}
\validMeasurement{\texttt{(input\_domain, input\_metric, output\_measure, \\plan, global\_scale, threshold)}}{\texttt{make\_private\_group\_by}}

\section{Proof}

We now prove the postcondition of \texttt{make\_private\_group\_by}.
\begin{proof}

Start by establishing properties of the following variables, 
which hold for any setting of the input arguments.

\begin{itemize}
    \item By the postcondition of \texttt{StableDslPlan.make\_stable}, \texttt{t\_prior} is a valid transformation. \ref{line:tprior}
    \item By the postcondition of \texttt{match\_grouping\_columns}, \texttt{grouping\_columns} holds the names of the grouping columns.
        \texttt{margin} denotes what is considered public information about the key set, 
        pulled from descriptors in the input domain. \ref{line:groupcols}
    \item By the postcondition of \texttt{make\_basic\_composition}, 
        \texttt{m\_exprs} is a valid measurement that prepares a batch of expressions that, 
        when executed, satisfies the privacy guarantee of \texttt{m\_exprs}.  \ref{line:joint}
\end{itemize}

We now reconcile information about the censoring threshold. \ref{line:reconcile-threshold}
In the setting where grouping keys are considered public, no thresholding is necessary.
In the setting where grouping keys are considered private information,
threshold information is prepared from either \texttt{predicate} or \texttt{threshold}.
By the post-condition of \texttt{find\_len\_expr}, filtering on \texttt{name} can be used to satisfy 
$\delta$-approximate DP.

The final predicate to be applied is the intersection of conditions necessary for filtering, 
and initial conditions set in the predicate.
The thresholding predicate is only added to the final predicate 
if the threshold is not already present in the predicate.

We now move on to the implementation of the function. \ref{line:function}
The function returns a DslPlan that applies each expression from \texttt{m\_exprs}
to \texttt{arg} grouped by \texttt{keys}.
\texttt{threshold\_info} is conveyed into the plan, if set, 
to ensure that the keys are also privatized if necessary.
It is assumed that the emitted DSL is executed in the same fashion as is done by Polars.
This proof/implementation does not take into consideration side-channels involved in the execution of the DSL.

We now move on to the implementation of the privacy map. \ref{line:privacy-map}
The measurement for each expression expects data set distances in terms of a triple:
\begin{itemize}
    \item $L^0$: the greatest number of partitions that can be influenced by any one individual. 
    This is no greater than the input distance (an individual can only ever influence as many partitions as they contribute rows), 
    but could be smaller when supplemented by 
    the \texttt{max\_influenced\_partitions} metric descriptor or \texttt{max\_num\_partitions} domain descriptor.
    \item $L^\infty$: the greatest number of records that can be added or removed by any one individual in each partition. 
    This is no greater than the input distance, 
    but could be tighter when supplemented by 
    the \texttt{max\_partition\_contributions} metric descriptor or the \texttt{max\_partition\_length} domain descriptor.
    \item $L^1$: the greatest total number of records that can be added or removed across all partitions.
    This is no greater than the input distance,
    but could be tighter when accounting for the $L^0$ and $L^\infty$ distances.
\end{itemize}

By the postcondition of the map on \texttt{m\_exprs}, the privacy loss, 
when grouped data sets may differ by this distance triple,
is \texttt{d\_out}.

% If the grouping keys are private, then the threshold has been prepared in \ref{reconcile-threshold}.
% At this point, it is necessary to show that the probability that any unstable partition is released is at most $\delta$.
% An unstable partition is any partition that may not exist on a neighboring data set:
% that is, an unstable partition is a partition whose data is unique to an individual.
% Therefore, unstable partitions have \texttt{li} records or fewer.

% The distance to instability (\texttt{d\_instability}) denotes the gap between the size of the largest unstable partition and the threshold.
% The additional delta parameter (\texttt{delta\_p}) denotes the worst-case probability 
% that an unstable partition an individual contributes is still present in the release,
% because noise added to release the DP count estimate exceeds \texttt{d\_instability}.

We now adapt the proof from \cite{rogers2023unifyingprivacyanalysisframework} (Theorem 7).
Consider $S$ to be the set of labels that are common between $x$ and $x'$.
Define event $E$ to be any potential outcome of the mechanism for which all labels are in $S$
(where only stable partitions are released).
We then lower bound the probability of the mechanism returning an event $E$.
In the following, $c_j$ denotes the exact count for partition $j$,
and $Z_j$ is a random variable distributed according to the distribution used to release a noisy count.

\begin{align*}
    \Pr[E] &= \prod_{j \in x \backslash x'} \Pr[c_j + Z_j \le T] \\
    &\ge \prod_{j \in x \backslash x'} \Pr[\Delta_\infty + Z_j \le T] \\
    &\ge \Pr[\Delta_\infty + Z_j \le T]^{\Delta_0}
\end{align*}

The probability of returning a set of stable partitions ($\Pr[E]$) 
is the probability of not returning any of the unstable partitions.
We now solve for the choice of threshold $T$ such that $\Pr[E] \ge 1 - \delta$.

\begin{align*}
    \Pr[\Delta_\infty + Z_j \le T]^{\Delta_0} &= \Pr[Z_j \le T - \Delta_\infty]^{\Delta_0} \\
    &= (1 - \Pr[Z_j > T - \Delta_\infty])^{\Delta_0}
\end{align*}

Let \texttt{d\_instability} denote the distance to instability of $T - \Delta_\infty$.
By the postcondition of \\ \texttt{integrate\_discrete\_noise\_tail},
the probability that a random noise sample exceeds \texttt{d\_instability} is at most \texttt{delta\_single}.
Therefore $\delta = 1 - (1 - \texttt{delta\_single})^{\Delta_0}$.
This privacy loss is then added to \texttt{d\_out}.

Together with the potential increase in delta for the release of the key set,
then it is shown that \function(x), \function(x') are \dout-close under \texttt{output\_measure}.

\end{proof}

\bibliographystyle{alpha}
\bibliography{make_private_group_by}
\end{document}